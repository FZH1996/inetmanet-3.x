\ifdraft TODO

\chapter{The UDP Model}
\label{cha:udp}

\section{The UDP module}

The state of the sockets are stored within the UDP module and the application
can configure the socket by sending command messages to the UDP module.
These command messages are distinguished by their kind and the type of their
control info. The control info identifies the socket and holds the parameters
of the command.

Applications don't have to send messages directly to the UDP module,
as they can use the \cppclass{UdpSocket} utility class, which encapsulates the messaging and
provides a socket like interface to applications.

\subsection{Sending UDP datagrams}

If the application want to send datagrams, it optionally can connect to the destination.
It does this be sending a message with UDP\_C\_CONNECT kind and \cppclass{UdpConnectCommand}
control info containing the remote address and port of the connection.
The UDP protocol is in fact connectionless, so it does not send any packets as a result
of the connect call. When the UDP module receives the connect request,
it simply remembers the destination address and port and use it as default destination
for later sends. The application can send several connect commands to the same socket.

% FIXME currently connect() or bind() is mandatory as the first command,
%       the application cannot send packets or set options otherwise

% FIXME connect() should allow unspecified dest address and -1 port (interpreted as disconnect())

For sending an UDP packet, the application should attach an \cppclass{UDPSendCommand}
control info to the packet, and send it to \nedtype{Udp}. The control info may contain
the destination address and port. If the destination address or port
is unspecified in the control info then the packet is sent to the connected target.

The \nedtype{Udp} module encapsulates the application's packet into an \msgtype{UDPPacket},
creates an appropriate IP control info and send it over ipOut or ipv6Out depending on
the destination address.

The destination address can be the IPv4 local broadcast address (255.255.255.255)
or a multicast address. Before sending broadcast messages, the socket must be configured
for broadcasting. This is done by sending an message to the UDP module. The message
kind is UDP\_C\_SETOPTION and its control info (an \cppclass{UdpSetBroadcastCommand})
tells if the broadcast is enabled. You can limit the multicast to the local network
by setting the TTL of the IP packets to 1. The TTL can be configured per socket,
by sending a message to the UDP with an \cppclass{UDPSetTimeToLive} control info
containing the value. If the node has multiple interfaces, the application can
choose which is used for multicast messages. This is also a socket option, the
id of the interface (as registered in the interface table) can be given in an
\cppclass{UdpSetMulticastInterfaceCommand} control info.

% FIXME currently sending broadcast messages is enabled without setting SO_BROADCAST to true,
%       this is not so in UNIX

% FIXME there should be a separate TTL for multicast (not used for unicast), default value is 1
%       see IP_MULTICAST_TTL in `man 7 ip`

\begin{note}
The \nedtype{Udp} module supports only local broadcasts (using the special 255.255.255.255 address).
Packages that are broadcasted to a remote subnet are handled as undeliverable messages.
\end{note}

If the UDP packet cannot be delivered because nobody listens on the destination port,
the application will receive a notification about the failure. The notification is
a message with UDP\_I\_ERROR kind having attached an \cppclass{UdpErrorIndication}
control info. The control info contains the local and destination address/port,
but not the original packet.

After the application finished using a socket, it should close it by sending a message
UDP\_C\_CLOSE kind and \cppclass{UdpCloseCommand} control info. The control info
contains only the socket identifier. This command frees the resources associated
with the given socket, for example its socket identifier or bound address/port.

\subsection{Receiving UDP datagrams}

Before receiving UDP datagrams applications should first ``bind'' to the given UDP port.
This can be done by sending a message with message kind UDP\_C\_BIND attached with an
\cppclass{UdpBindCommand} control info. The control info contains the socket identifier
and the local address and port the application want to receive UDP packets.
Both the address and port is optional. If the address is unspecified, than the UDP
packets with any destination address is passed to the application. If the port is
-1, then an unused port is selected automatically by the UDP module.
The localAddress/localPort combination must be unique.

When a packet arrives from the network, first its error bit is checked. Erronous messages
are dropped by the UDP component. Otherwise the application bound to the destination port
is looked up, and the decapsulated packet passed to it. If no application is bound to
the destination port, an ICMP error is sent to the source of the packet. If the socket is
connected, then only those packets are delivered to the application, that received from
the connected remote address and port.

The control info of the decapsulated packet is an \cppclass{UDPDataIndication}
and contains information about the source and destination address/port, the TTL,
and the identifier of the interface card on which the packet was received.

The applications are bound to the unspecified local address, then they receive any packets
targeted to their port. UDP also supports multicast and broadcast addresses; if they
are used as destination address, all nodes in the multicast group or subnet receives the packet.
The socket receives the broadcast packets only if it is configured for broadcast.
To receive multicast messages, the socket must join to the group of the multicast address.
This is done be sending the UDP module an UDP\_C\_SETOPTION message with
\cppclass{UdpJoinMulticastGroupsCommand} control info. The control info specifies the
multicast addresses and the interface identifiers. If the interface identifier is given
only those multicast packets are received that arrived at that interface.
The socket can stop receiving multicast messages if it leaves the multicast group.
For this purpose the application should send the UDP another UDP\_C\_SETOPTION
message in their control info (\cppclass{UdpLeaveMulticastGroupsCommand}) specifying
the multicast addresses of the groups.

% TODO clarify: multicast packets should not be delivered to connected sockets?

\subsection{Signals}

The \nedtype{Udp} module emits the following signals:
\begin{itemize}
  \item \fsignal{sentPk} when an UDP packet sent to the IP, the packet
  \item \fsignal{rcvdPk} when an UDP packet received from the IP, the packet
  \item \fsignal{passedUpPk} when a packet passed up to the application, the packet
  \item \fsignal{droppedPkWrongPort} when an undeliverable UDP packet received, the packet
  \item \fsignal{droppedPkBadChecksum} when an erronous UDP packet received, the packet
\end{itemize}

\fi


